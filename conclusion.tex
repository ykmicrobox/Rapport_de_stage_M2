\section{Conclusion}
\addetoc{section}{Conclusion}
\paragraph{}
Le but de mon stage était de mettre en \oe{}uvre un module de spécialisation
\emph{OpenCL}, en utilisant l'infrastructure de compilation LLVM et en ce basant
sur un modèle et une architecture logicielle définit par les précédente
passes déjè développées, prenant en entrée un code écrit en représentation
intermédiaire générique, et produit un code optimisé pour les périphériques de
calcul OpenCL, tels que le Xeon Phi.

J’ai rempli cet objectif en fournissant ce module. J'ai aussi effectué les
modifications et implémentations nécessaires dans le runtime pour lui permettre
de supporter des modules écrits en SPIR, car avant ce stage, nous ne savions pas
encore si cela était possible.

Ce travail bien qu'il reste à y effectuer quelques optimisations pour la
production de code plus performant, a été jugé de qualité suffisante pour être
intégré au reste de la chaîne de compilation et servir de cas démonstrateur au
près des partenaire du projet.

Le projet \emph{Mach} peut maintenant cibler une machine hétérogène. Néanmoins,
le système actuel n'est pas prévu pour fonctionner sur une machines distribuées,
ce qui est nécessaires à l'exploitation correct des calculateurs utilisés dans
le domaine du HPC. Le projet \emph{$M^2DC$ (H2020)} en cours de développement dans
lequel intervient \emph{AS+} auprès d'autres partenaires vient conforté ce
système en fournissant un runtime permettant de gérer dynamiquement
l'ordonnancement des tâches sur plusieurs n\oe{}uds de calculs.

Pour finir, ce stage m'a permit d'apporter ma contribution à un projet
professionnel européen, et mettre en profit mes connaissances en HPC acquissent
tout au long de mon cursus universitaire, et d'approfondir mes compétences dans
ce domaine.
