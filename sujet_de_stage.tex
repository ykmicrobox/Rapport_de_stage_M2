\section{Présentation du sujet de stage}
\addetoc{section}{Presentation of the internship}
\subsection{Descriptif général du projet Mach}
\addetoc{subsection}{General description of the Mach project}
\paragraph{ITEA Mach}
est un projet européen regroupant une vingtaine de partenaires dont AS+, le
CEA-LIST, l'INRA, Silkan, Thales Communications and security... L'objectif de ce
projet est de faciliter la programmation sur machines hybrides combinant CPU et
accélérateurs tels que GPU et Xeon-Phi, en partant d'une approche de type DSL
(Domain Specific Language) embarqués si besoin dans un langage hôte (DSeL). Pour
valider les frameworks mis en place, le projet dispose d’un certain nombre de
cas utilisateurs représentatifs de différents domaines : traitement d’image, du
signal, mécanique des fluides, biostatistiques, etc.

\paragraph{}
Le rôle d'AS+ est de fournir les technologies permettant de construire des
chaines de compilation modulaire construite autour du compilateur LLVM et un
modèle d’exécution de type tâche pour des architectures hétérogènes en
s’appuyant sur les DSL mis en place pour chaque domaine. La Société travaille
plus particulièrement avec l’INRA et le CEA-LIST avec le langage R dans le
domaine des biostatistiques. L’INRA apporte les applications et le CEA-LIST
apporte la traduction du langage R vers notre DSL.

\paragraph{}
On peut décrire la chaine de compilation développée par AS+ comme ceci : On
prend en entrée du code dans une version étendue du langage LLVM-IR, annotée si
besoin pour faciliter la compilation. Le code passe par le tâchifieur qui
s'occupe de regrouper en tâches les instructions réalisées et de les placer dans
des procédures. Une fois le programme découpé en tâches, les codes passent par
les différents spécialisateurs pour être transformés en code LLVM-IR pur, avec
les annotations adaptées à une architecture spécifique. Ils peuvent ensuite
passer dans la chaine de compilation standard de LLVM pour être optimisés et
compilés sous forme de binaire. La gestion de la mémoire et les appels aux
bibliothèques sont transformés en équivalents fournis par le runtime. Le tout
peut ensuite être regroupé dans un exécutable capable de tourner sur une machine
hétérogène.

\subsection{Runtime}
\addetoc{subsection}{Runtime}
\subsubsection{Mach Runtime}
\addetoc{subsubsection}{Mach Runtime}
\paragraph{}
Nous avons donc besoin d'un runtime hétérogène, qui prendra en charge une
application conçue comme un ensemble de tâches, nous voulons donc lui laisser le
choix de gérer quand et dans quel ordre il va les exécuter, sachant qu'il
connait les accès mémoires de chaque tâche, il pourra par conséquent se
construire un graphe de dépendance de données entre ces dernières, ainsi tout en
respectant un tant soit peu l'ordre dans lequel les tâches lui sont données, le
runtime peut les réordonner quand cela permet une exécution parallèle.\\
Le découpage en tâche de l’application permet une répartition de charge
dynamique sur les différentes ressources de calcul indépendamment de
l’application. Il doit donc être capable de déterminer les unités de calcul
disponibles. On lui incombe aussi la gestion de la mémoire et de s’occupe des
allocations et des copies mémoires dans le cas où des unités de calcul accèdent
à de la mémoire différente (comme le CPU et un GPU ou Xeon-Phi).

Il existe des runtimes qui répondent à nos attentes, nous allons donc en
utiliser un, StarPU. Mais nous allons tout de même définir une interface
d'abstraction pour limiter l'adhérence au runtim sous-jacent et nous permettre
d'ajouter des fonctionnalités absentes de ce dernier sans avoir à le modifier en
profondeur.

\subsubsection{STARPU}
\addetoc{subsubsection}{STARPU}
\paragraph{}
StarPU est un runtime par tâches pour systèmes multic\oe{}urs hétérogène
développé par l'équipe runtime de l'INRIA de Bordeaux.

Il peut gérer les systèmes multi-processeurs, GPU Nvidia en utilisant Cuda et
les différents périphériques répondant à OpenCL (GPU AMD ou Intel Xeon-Phi par
exemple).\\
Le runtime intègre également un gestionnaire de mémoire. Il est capable de gérer
les dépendances entre les tâches à l’aide des intentions fournies avec
celles-là. Il fournit également la possibilité de découper les zones mémoires
enregistrées en sous parties, pour pouvoir par exemple permettre à des tâches
accédant à des zones différentes de la donnée de tourner en parallèle.

\subsection{LLVM}
\addetoc{subsection}{LLVM}
\subsubsection{Déscription}
\addetoc{subsubsection}{Description}
\paragraph{}
Anciennement appelé \og{} Low Level Virtual Machine \fg{} (Machine Virtuelle de
Bas Niveau en français), \emph{LLVM} est un projet d'infrastructure de
compilation qui permet l'optimisation de code indépendamment de toute
plate-forme et de tout langage à différentes étapes de la compilation ou à
l'exécution du programme avec un compilateur à la volée.\\
Le projet a débuté en 2000 en tant que projet de recherche sous la direction de
Vikram Adve et Chris Lattner à l'Université de l'Illinois à Urbana–Champaign. Il
est à présent distribué sous forme de logiciel libre avec une licence permissive
et il est écrit sous forme de bibliothèques réutilisables.

Il définit une séparation claire des phases de compilation, un langage
intermédiaire et un ensemble de back-end pouvant générer du code pour une
architecture spécifique.\\
Il propose aussi depuis sa version 2.6 le compilateur Clang pour les langages C,
C++, Objective-C et Objective-C++.\\
Ce frontale se charge de traduire le code en langage intermédiaire de plus bas
niveau, le c\oe{}ur de LLVM appliquera ensuite des passes de transformation et
d'optimisation de code, et le back-end le code en assembleur pour une
architecture spécifique.

\paragraph{}
La modularité est un mot clés dans LLVM, grâce à cette architecture, l'ajout de
la prise en charge d'un nouveau langage de haut niveau est réduit à écrire le
frontale de traduction vers le LLVM-IR et la prise en charge d'un nouveau
processeur avec le back-end correspondant. Mais LLVM impose cette séparation des
étapes de manière formelle : le seul outil de communication entre les étapes est
le langage LLVM-IR.

\subsubsection{Pass}
\addetoc{subsubsection}{Pass}
\paragraph{}
La partie centrale fonctionne par l'application d’un ensemble de passes. Il y a
deux types de passes dans LLVM, les passes d’analyse, qui génèrent des
statistiques utilisables par d’autres passes sans modification du code, et les
passes de transformations, qui modifie le code principalement pour l'optimiser
en se basant potentiellement sur les statistiques générée par le premier type de
passe. On peut indiquer des dépendances entre les passes. Celles-ci peuvent
nécessiter le résultat d’une autre passe ou bien invalider le travail d’une
autre, qui sera à exécuter à nouveau.

\subsubsection{IR}
\addetoc{subsubsection}{IR}
\paragraph{}
Le langage intermédiaire utilisé par LLVM est un langage d'assemblage sous forme
SSA (Static Single Assignment), c'est-à-dire qu'on ne peut affecter une valeur à
un registre qu'une seul fois, c'est un langage fortement typée et possède un
nombre infini de registres.

Ce langage peut être utilisé sous trois formats de représentation : la
représentation mémoire pour être utilisée les outils de compilation ; le format
bitcode, pour être stocker sur disque ; et le format assembleur pour être
humainement lisible.

\subsection{IR Étendu}
\addetoc{subsection}{Extended IR}
\paragraph{}
En dépit de tout ce que nous offre le langage LLVM IR, il manque certaines
caractéristiques et manipulation que nous allons énoncer.

\subsubsection{ALV (Vecteur de taille arbitraire)}
\addetoc{subsubsection}{ALV (Arbitrary Lenght Vector)}
\paragraph{}
L'intérêt de ces vecteurs de tailles arbitraire est de simplifier la compilation
en laissant au module de spécialisation la soin de faire tenir l'instruction
dans l'unité vectorielle gérée par le matériel ciblé sans avoir à préciser de
taille au préalable, étant donné que la majorité des architectures matérielles
sont de type SIMD(Single Instruction Multiple Data)/SIMT(Single Instruction
Multiple Thread), une même instruction est donc appliqué sur un vecteur de
données en parallèle, ou a un sous-ensemble de vecteur avec masquage.\\
Les ALV sont marqués de taille 0 en IR. Ils sont utiles pour gérer les données
spécifiques lors des instructions de chargement/stockage (load/store). Toute les
opération sur les vecteurs classiques sont supportées sur les ALV.

\subsubsection{ND array}
\addetoc{subsubsection}{ND array}
\paragraph{}
Ces tableaux à \emph{n} dimensions sont inspirés de la bibliothèque \emph{NumPy}
de \emph{Python}. Ces conteneurs sont transparents pour chaque types de données.
Ils offrent un concept de vue, c'est-à-dire que plusieurs ND array peuvent se
partager les mêmes données mais y travailler et accéder de manières différentes.
Des fonctions de manipulation sont donc implémenter dans le runtime pour gérer
la copie, l'extraction etc.\ de données de ces conteneurs.

\subsubsection{Marquer d'Intents}
\addetoc{subsubsection}{Intents marker}
\paragraph{}
On les ajoute dans la déclaration d'une tâche, il permettent de définir le type
d'accès sur un tableaux : lecture seule, écriture seule, lecture/écriture ou
Scratchpad. Ce dernier est un cas particulier de vecteurs, on l'insère comme
premier argument dans toutes les tâches spécialisées, il nous permet de stocker
certaines données telle que la taille des dimensions des ND array, ou le
décalage à effectuer pour la vue. Il sert aussi de stockage temporaire des
résultats partielles lors d'une réduction parallèle.

\subsubsection{Fonctions particulières}
\addetoc{subsubsection}{Special Function}
\paragraph{}
On définit trois types de tâches en se basant sur une autres notions de la
bibliothèque \emph{NumPy}, les \emph{ufunc}, \emph{rfunc} et \emph{sfunc}. Cette
information est passée dans les métadonnées pour exprimer des concepts sur la
mise en \oe{}uvre des tâches.

\emph{ufunc} (Universal function), est une fonction opérant sur des ND array
élément par élément.

\emph{rfunc} (Reduce function), est une opération de réduction s'appliquant à
tous les éléments d'un vecteur. Le ND array en entrée peut être directement le
résultat d'une fonction optionnelle \emph{ufunc}.

\emph{sfunc} (Scan function), est une opération de somme cumulative sur un ND
array.

\subsection{Spécialisateur CUDA}
\addetoc{subsection}{Cuda Specializer}

\subsection{KFE}
\addetoc{subsection}{KFE}
