\section{Présentation de l'entreprise et du service}
\addetoc{section}{Company and service presentation}
\paragraph{AS+}
est une société de conseil et d'ingénierie filiale du groupe \emph{EOLEN} depuis
2012. Ce dernier est présent dans différents secteurs tels que les
télécommunications, la finance, l'aéronautique, le spatial, la défense,
l'énergie et les sciences de la vie. Le groupe est principalement implanté en
\emph{Ile-de-France} mais compte aussi une implantation d'une cinquantaine de
personnes au \emph{Brésil}. Le groupe a réalisé plus de 23M\euro{} de chiffre
d'affaire en 2015 et compte environ 300 personnes.

\paragraph{}
AS+ est spécialisé dans les métiers des télécommunications, l'informatique
scientifique et industriel et dans le calcul intensif (HPC).

\paragraph{}
Le bureau d’étude HPC, fort d'une vingtaine de collaborateurs, a développé
depuis plusieurs années une forte expertise sur les méthodes et outils de
développement dédiés aux plates-formes de calcul intensif : architectures
multic\oe{}urs, accélérateurs de type GPU et many-core (Xoen-Phi), clusters de
calcul et a bâti une offre de services complète portant sur le développement,
l’optimisation et le portage sur architectures parallèles de codes de calcul et
suivant des modes d’intervention au plus proche des besoins des clients :
conseil/audit, formations, assistance technique ou prestations clé en main. AS+
compte parmi ses clients les principaux acteurs du domaine en France comme le
\emph{CEA}, \emph{TOTAL}, l’\emph{ONERA}, l’\emph{IDRIS}, l’\emph{ANDRA},
l’\emph{IRSN}, \emph{ATOS/Bull}, \emph{IBM}, \emph{DAHER}.

\paragraph{}
\sloppy
L’équipe HPC intervient également très en amont dans l’écosystème du calcul
intensif aux côtés de partenaires industriels et académiques tels que
l’\emph{INRA}, le \emph{CEA}, \emph{THALES} et ce dans le cadre de projets R\&D
tels que \emph{OpenGPU (FUI)}, \emph{Brainomics (FSN)} \emph{M2DC (H2020)} ou
\emph{MACH (ITEA)}. \emph{AS+} propose également un catalogue de formations
spécifiques au calcul intensif qui comprend notamment des modules dédiés aux
technologies \emph{CUDA}, \emph{OpenCL}, \emph{MPI/OpenMP} ou aux suites
\emph{Intel Parallel Studio / Cluster Studio}.

\paragraph{}
\emph{AS+} est par ailleurs membre de l’association \emph{Ter@tec} depuis 2012
qui rassemble les principaux acteurs académiques et industriels du calcul
intensif en France. La société a, depuis, concrétisé un certain nombre d’actions
en lien direct avec l’association, notamment en participant depuis cinq ans au
forum TERATEC, et la mise en place d'une démarche de partenariats avec d’autres
membres de l’association, tels que \emph{Activeeon}, \emph{Caps Entreprises},
\emph{Nvidia} et \emph{Intel}. De plus la société est installée depuis octobre
2012 sur le Campus \emph{Ter@tec} à Bruyères le Châtel. Cela permet de renforcer
la synergie entre l'équipe R\&D et celle déjà présente au \emph{CEA} et au
\emph{TGCC} (Très Grand Centre de Calcul) qui a la responsabilité du support
applicatif aux utilisateurs du centre de calcul.
