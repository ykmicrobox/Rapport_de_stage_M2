\section{Introduction}
\addetoc{section}{Introduction}
\paragraph{}
On m'a attribué dans le cadre de ce stage que j'ai effectué au sein
d'\emph{AS+}, une filiale du groupe \emph{EOLEN}, la mission de développement
d'un module de spécialisation \emph{OpenCL} en utilisant l'infrastructure de
compilation \emph{LLVM} devant s'intégrer à une chaîne de compilation et
d’exécution pour des architectures hétérogènes et s’appuyant sur les DSL (Domain
Specific Language ou \emph{langages dédiés} en français) mis en place.

Ce travail est inscrit dans le cadre du projet \emph{ITEA Mach}. C'est un projet
européen regroupant une vingtaine de partenaires, dont le \emph{CEA LIST} et
l'\emph{INRA} deux entreprises avec qui travaille plus particulièrement
\emph{AS+} dans le domaine des biostatistiques avec le langage R comme DSL.

\paragraph{}
J’ai intégré l’équipe projet \emph{AS+}, constituée d’un ingénieur d’études
senior -- Gaétan \textsc{Bayle des Courchamps} -- responsable du développement de
l’interface avec le runtime tâche, de deux experts HPC -- Vincent \textsc{Ducrot}
et Kevin \textsc{Juilly} -- en charge des spécifications et de la conception de
la chaîne de compilation et d’exécution, et d’un chef de projet -- Sébastien
\textsc{Monot} -- responsable technique de l’activité HPC et du pilotage des
projets en particulier.

Le projet est en phase finale de développement, j'ai par conséquent basé mon
travail sur un modèle et une architecture logicielle déjà définit, pour
l'intégrer au plus simple au reste de la chaîne.

\paragraph{}
Dans les sections suivantes, je commence par présenter plus en détail
l'entreprise d'accueil et le service auquel je suis affecté, suivra la
présentation du projet \emph{Mach} sur lequel je travail et les logiciels
utilisés comme fondation à son développement. Je poursuivrai avec le déroulement
du travail que j'ai effectué tout au long de ma période de stage, la prise en
main des outils au développement du prototype de spécialisateur et d’autres
éléments de la chaine de compilation. Je montrerai ensuite les résultats obtenus
par ce que j’ai réalisé. Je finirai par une conclusion résumant le travail
effectué et les perspectives attendu de ce projet.
